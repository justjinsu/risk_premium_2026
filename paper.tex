\documentclass[preprint,12pt]{elsarticle}

\usepackage{amssymb}
\usepackage{amsmath}
\usepackage{booktabs}
\usepackage{graphicx}
\usepackage{hyperref}
\usepackage{geometry}
\usepackage{multirow}
\usepackage{array}
\usepackage{xcolor}
\usepackage{float}
\usepackage{threeparttable}
\usepackage[utf8]{inputenc}
\geometry{a4paper, margin=1in}

\journal{Energy Policy}

\begin{document}

\begin{frontmatter}

\title{Quantifying the Climate Risk Premium: A Case Study of the Samcheok Blue Power Plant in South Korea}

\author[1]{Jinsu Park\corref{cor1}}
\ead{jinsu@planit.institute}

\address[1]{PLANiT Institute, Seoul, South Korea}

\cortext[cor1]{Corresponding author}

\begin{abstract}
As the global energy transition accelerates, coal-fired power assets face increasing stranded asset risks from both policy constraints and physical climate hazards. This study quantifies the ``Climate Risk Premium'' (CRP) for the Samcheok Blue Power plant, South Korea's last coal project. We integrate three data sources into a unified financial model: (1) Korea's 10th National Power Supply Plan coal dispatch trajectories (2024-2036), (2) CLIMADA climate hazard data for wildfire (Fire Weather Index), flood (GLOFAS), and sea level rise (IPCC AR6), and (3) Korea Investors Service (KIS) credit rating methodology with six quantitative metrics. By linking dispatch reductions and climate hazards directly to cash flows and financing costs---including corporate tax (24\%), straight-line depreciation, and debt service schedules---we demonstrate that transition scenarios reduce the baseline NPV of \$2.9 billion by 251--375\%, while physical climate risks reduce NPV by 33--59\%. Combined transition and physical risks trigger a ``credit rating death spiral'' where falling Debt Service Coverage Ratios (DSCR from 1.81$\times$ to negative values) cause rating downgrades from investment grade (BBB) to speculative grade (B), increasing spreads from 250 to 600 basis points. The total Climate Risk Premium reaches 5,854 basis points under aggressive scenarios, rendering the project economically unviable. These findings demonstrate that government policy---not physical climate change---is the primary driver of coal asset stranding in Korea, and highlight the urgent need for structured just transition finance mechanisms to manage orderly coal phase-out.
\end{abstract}

\begin{keyword}
Climate Risk Premium \sep Stranded Assets \sep Coal Phase-out \sep Project Finance \sep South Korea
\end{keyword}

\end{frontmatter}

\section{Introduction}

\subsection{The Samcheok Paradox}
South Korea faces a critical dilemma in its energy transition. While committing to carbon neutrality by 2050 and announcing accelerated coal phase-out targets in its 10th Basic Plan for Electricity Supply and Demand (2023-2036), the country recently commissioned the 2.1 GW Samcheok Blue Power plant in 2024—likely the last coal-fired power plant in its history. This contradiction presents a unique case study for analyzing "stranded asset" risk in real-time.

\subsection{Research Gap}
Traditional financial models often underestimate climate risks by treating them as exogenous shocks or simple scenario parameters. Few studies integrate \textit{actual government policy trajectories} (such as Korea's power supply plan dispatch reductions) with \textit{spatially-explicit physical hazard data} (such as CLIMADA wildfire, flood, and sea level rise projections) into a unified financial framework that explicitly models \textit{credit rating migration} and cost of capital feedbacks.

\subsection{Contribution}
This paper proposes an integrated framework that links transition and physical risks directly to financial performance and, crucially, to the cost of capital. We make three key contributions:

\begin{enumerate}
    \item \textbf{Korea Power Supply Plan Integration}: We incorporate official government coal dispatch trajectories from the 10th Power Supply Plan, translating national policy into plant-level capacity factor reductions and revenue impacts.

    \item \textbf{CLIMADA Physical Risk Quantification}: We apply CLIMADA (Climate Adaptation Platform) hazard data to quantify wildfire, flood, and sea level rise impacts on the Samcheok plant, moving beyond generic physical risk assumptions to location-specific climate science.

    \item \textbf{Credit Rating Death Spiral}: We introduce the concept of the "Climate Risk Premium" (CRP)—the additional yield investors demand when climate risks trigger credit rating downgrades—and demonstrate how policy-driven dispatch reductions and climate hazards create a non-linear feedback loop that accelerates asset stranding.
\end{enumerate}

\section{Theoretical Framework}

We develop a bottom-up valuation model that explicitly incorporates climate risk vectors into both the cash flow numerator and the discount rate denominator.

\subsection{Integrated Cash Flow Model}
The project's Net Present Value (NPV) is defined as:
\begin{equation}
    NPV = \sum_{t=1}^{T} \frac{CF_t}{(1 + WACC)^t} - I_0
\end{equation}
where $CF_t$ is the free cash flow in year $t$, $WACC$ is the weighted average cost of capital, and $I_0$ is the initial investment.

We modify the standard cash flow function to account for climate risks:
\begin{equation}
    CF_t = (EBIT_t \cdot (1 - \tau)) + Depr_t - Capex_t - \Delta WC_t
\end{equation}
where $\tau$ is the corporate tax rate (24\%) and $Depr_t$ is straight-line depreciation over 30 years.

\subsubsection{Physical Risk Adjustments}
Physical risks (drought, heat, wildfire) reduce the effective generation quantity $Q_t^{gen}$. We model this as a constraint on the capacity factor $\kappa$:
\begin{equation}
    Q_t^{gen} = Cap \cdot 8760 \cdot \min(\kappa_{base}, \kappa_{water}, (1 - \delta_{outage}))
\end{equation}
where $\kappa_{water}$ represents the maximum capacity factor sustainable under water availability constraints (e.g., 50\% during severe drought), and $\delta_{outage}$ is the forced outage rate due to wildfires.

\subsubsection{Transition Risk Adjustments}
Transition risks primarily impact costs via carbon pricing. The carbon cost $C_t^{carbon}$ is given by:
\begin{equation}
    C_t^{carbon} = Q_t^{gen} \cdot E_{intensity} \cdot P_t^{CO2}
\end{equation}
where $P_t^{CO2}$ follows the NGFS Net Zero trajectory, rising significantly over time.

\subsection{The Credit Rating Death Spiral}
A key contribution of this paper is the endogenous modeling of the cost of debt. Unlike static models, we define the cost of debt $r_d$ as a function of the project's credit rating $R_t$, which in turn depends on the Debt Service Coverage Ratio ($DSCR_t$):

\begin{equation}
    DSCR_t = \frac{CF_t^{avail}}{DebtService_t}
\end{equation}

The credit rating $R_t$ is determined by a step function mapping $DSCR_t$ to rating notches (based on KIS methodology):
\begin{equation}
    R_t = f(DSCR_t) \in \{AAA, AA, ..., B, CCC\}
\end{equation}

The cost of debt is then:
\begin{equation}
    r_d(t) = r_f + \text{Spread}(R_t) + \text{CRP}
\end{equation}

As climate risks reduce $CF_t$, $DSCR_t$ falls. If it breaches critical thresholds (e.g., 1.2$\times$), the rating $R_t$ downgrades, causing $\text{Spread}(R_t)$ to spike. This increases interest expense, further lowering $DSCR_t$---a feedback loop we term the ``Credit Rating Death Spiral,'' illustrated in Figure~\ref{fig:death_spiral}.

\begin{figure}[H]
    \centering
    \includegraphics[width=0.95\textwidth]{data/processed/figures/fig5_death_spiral.png}
    \caption{The Credit Rating Death Spiral. Climate risks reduce revenue and cash flows, which lowers DSCR and triggers credit rating downgrades. Lower ratings increase cost of debt (spread widens), which further reduces cash flows through higher interest expense---creating a self-reinforcing feedback loop. Once DSCR falls below 1.0$\times$, the project cannot service debt, leading to technical default.}
    \label{fig:death_spiral}
\end{figure}

\section{Methodology \& Data}

Our methodology integrates three independent data sources into a unified financial model, as shown in Figure~\ref{fig:data_integration}. Each data source provides distinct but complementary information: the Korea Power Supply Plan defines transition risk trajectories, CLIMADA provides spatially-explicit physical hazard data, and the KIS methodology maps financial metrics to credit ratings and spreads.

\begin{figure}[H]
    \centering
    \includegraphics[width=0.9\textwidth]{data/processed/figures/fig6_data_integration.png}
    \caption{Data Integration Framework. Three independent data sources---Korea Power Supply Plan (MOTIE), CLIMADA hazard data (ETH Zurich), and KIS credit methodology---feed into the CRP Model Integration Layer, which produces NPV, IRR, DSCR, Credit Rating, and Climate Risk Premium outputs.}
    \label{fig:data_integration}
\end{figure}

\subsection{Plant Specifications}
The analysis is based on public specifications of the Samcheok Blue Power plant:
\begin{itemize}
    \item \textbf{Capacity}: 2,100 MW (2 × 1,050 MW units)
    \item \textbf{Technology}: Ultra-Supercritical (USC) coal-fired steam turbines
    \item \textbf{Location}: Samcheok City, Gangwon Province (37.44°N, 129.17°E)
    \item \textbf{Investment}: 4.9 trillion KRW (\$4.9 billion, 2024 USD)
    \item \textbf{Commercial Operation}: Unit 1 (May 2024), Unit 2 (October 2024)
    \item \textbf{Design Life}: 30 years (standard for Korean coal plants)
    \item \textbf{Financing}: 70\% debt / 30\% equity, corporate bonds at 6.1-7.4\% yields
\end{itemize}

\subsection{Data Sources}

We obtain coal dispatch trajectories from the Ministry of Trade, Industry and Energy's 10th Basic Plan for Electricity Supply and Demand (2023-2036):

\begin{itemize}
    \item \textbf{2024 Baseline}: Coal generation = 195 TWh (32.5\% of total demand)
    \item \textbf{2030 NDC Target}: Coal generation = 130 TWh (19.3\%), capacity factor = 45\%
    \item \textbf{2036 Plan Endpoint}: Coal generation = 95 TWh (12.9\%), capacity factor = 32\%
    \item \textbf{2050 Net-Zero}: Coal generation < 15 TWh (1.7\%), capacity factor = 4\%
\end{itemize}

For Samcheok specifically (2.1 GW of 42 GW total Korean coal capacity = 5\%), we allocate proportional generation and back-calculate implied capacity factors. This approach directly translates national policy into plant-level financial impacts, ensuring our analysis reflects \textit{actual government commitments} rather than hypothetical scenarios.

\subsubsection{Physical Hazard Data}
We quantify three climate risks at the Samcheok location (37.44°N, 129.17°E) using data from established climate databases:

\textbf{1. Wildfire Risk}
\begin{itemize}
    \item \textbf{Data}: Fire Weather Index (FWI) from ERA5-Land reanalysis and Korean Meteorological Administration (KMA) Climate Change Assessment Report 2020
    \item \textbf{Baseline (1991-2020)}: FWI = 18.5 (1.2\% baseline transmission outage rate)
    \item \textbf{RCP 4.5 (2050)}: FWI = 26.2 (+42\%), outage rate = 1.7\%/year
    \item \textbf{RCP 8.5 (2050)}: FWI = 35.8 (+94\%), outage rate = 2.2\%/year
    \item \textbf{Mechanism}: Increased wildfire frequency in mountainous transmission corridors (Samcheok is 120 km from grid through fire-prone areas, with 2022 Uljin-Samcheok mega-fire precedent burning 213 km²)
\end{itemize}

\textbf{2. Flood Risk (Riverine + Coastal)}
\begin{itemize}
    \item \textbf{Data}: GLOFAS v3.1 (Global Flood Awareness System) reanalysis and K-water flood hazard maps for Gangwon Province
    \item \textbf{Baseline}: 1-in-10 year flood depth = 1.8m, 1-in-100 year = 4.2m; annual outage rate = 0.14\%
    \item \textbf{RCP 8.5 (2050)}: Monsoon intensification multiplier = 1.75$\times$; annual outage rate = 0.24\%
    \item \textbf{Mechanism}: Plant is 2-3 km from East Sea, cooling water intake at 5m elevation, Osip Creek riverine exposure creates compound flood risk
\end{itemize}

\textbf{3. Sea Level Rise}
\begin{itemize}
    \item \textbf{Data}: IPCC AR6 WG1 Chapter 9 regional projections for East Asian Marginal Seas (Table 9.9), relative to 1995-2014 baseline
    \item \textbf{RCP 4.5 (2050)}: +0.24m $\rightarrow$ 2.4\% capacity derate
    \item \textbf{RCP 8.5 (2050)}: +0.32m $\rightarrow$ 3.2\% capacity derate
    \item \textbf{RCP 8.5 (2100)}: +0.73m (median; range 0.52-1.01m)
    \item \textbf{Mechanism}: Reduced hydraulic head for cooling water pumps, amplified storm surge vulnerability
\end{itemize}

\subsubsection{Korea Investors Service (KIS) Credit Rating Methodology}
We map financial performance to credit ratings using KIS's quantitative grid for the Private Power Generation (IPP) sector. The methodology uses six metrics across three categories, as shown in Table~\ref{tab:kis_criteria}.

\begin{table}[H]
\centering
\caption{KIS Credit Rating Quantitative Criteria for Power Generation}
\label{tab:kis_criteria}
\small
\begin{tabular}{lcccccc}
\toprule
\textbf{Metric} & \textbf{AAA} & \textbf{AA} & \textbf{A} & \textbf{BBB} & \textbf{BB} & \textbf{B} \\
\midrule
\multicolumn{7}{l}{\textit{Business Stability}} \\
\quad Capacity (MW) & $\geq$2,000 & $\geq$800 & $\geq$400 & $\geq$100 & $\geq$20 & $<$20 \\
\quad EBITDA/Fixed Assets (\%) & $\geq$15 & $\geq$11 & $\geq$8 & $\geq$4 & $\geq$1 & $<$1 \\
\midrule
\multicolumn{7}{l}{\textit{Coverage Ratios}} \\
\quad EBITDA/Interest ($\times$) & $\geq$12 & $\geq$6 & $\geq$4 & $\geq$2 & $\geq$1 & $<$1 \\
\midrule
\multicolumn{7}{l}{\textit{Leverage Ratios}} \\
\quad Net Debt/EBITDA ($\times$) & $\leq$1 & $\leq$4 & $\leq$7 & $\leq$10 & $\leq$12 & $>$12 \\
\quad Debt/Equity (\%) & $\leq$80 & $\leq$150 & $\leq$250 & $\leq$300 & $\leq$400 & $>$400 \\
\quad Debt/Assets (\%) & $\leq$20 & $\leq$40 & $\leq$60 & $\leq$80 & $\leq$90 & $>$90 \\
\midrule
\textbf{Spread (bps)} & 50 & 100 & 150 & 250 & 400 & 600 \\
\bottomrule
\end{tabular}
\end{table}

The overall rating follows a conservative methodology where the worst-performing component drives the final assessment. Investment grade threshold lies at BBB/BB boundary (250/400 bps spread). This allows us to endogenize cost of debt as a function of climate-driven financial deterioration, creating the feedback mechanism central to our ``death spiral'' analysis.

\subsection{Scenario Design}

We simulate seven scenarios combining Korea Power Plan trajectories, CLIMADA hazards, and carbon pricing:

\begin{table}[h]
\centering
\caption{Scenario Matrix}
\begin{tabular}{lcccc}
\toprule
Scenario & Power Plan & CLIMADA & Carbon Price & Early Retirement \\
\midrule
Baseline & None & None & \$0 & None \\
10th Plan Only & Official & None & \$15-80 & 2054 \\
CLIMADA Only & None & RCP 8.5 & \$0 & None \\
Moderate Combined & Official & RCP 4.5 & \$15-100 & 2054 \\
Aggressive Combined & Accelerated & RCP 8.5 & \$25-200 & 2049 \\
\bottomrule
\end{tabular}
\end{table}

This design isolates the marginal contribution of each risk source (policy dispatch vs physical hazards vs carbon pricing) and demonstrates their interaction effects.

\section{Results}

\subsection{Financial Impact of Climate Risks}

Table 2 summarizes the Net Present Value (NPV), credit rating migration, and Climate Risk Premium across nine scenarios.
 
 \begin{figure}[h]
     \centering
     \includegraphics[width=1.0\textwidth]{data/processed/figures/fig1_npv_comparison.png}
     \caption{Impact of Policy and Physical Risks on Project NPV. Transition risks (red) drive deep negative value, while physical risks (orange) erode value further.}
     \label{fig:npv_comparison}
 \end{figure}

\begin{table}[H]
\centering
\caption{Comprehensive Scenario Analysis Results}
\label{tab:scenario_results}
\small
\begin{tabular}{lrrcrr}
\toprule
\textbf{Scenario} & \textbf{NPV (\$M)} & \textbf{$\Delta$ NPV} & \textbf{Min DSCR} & \textbf{Rating} & \textbf{CRP (bps)} \\
\midrule
Baseline (No constraints) & 2,898 & --- & 1.81$\times$ & BBB & --- \\
\midrule
\multicolumn{6}{l}{\textit{Transition Risk Only (Korea Power Plan)}} \\
\quad Moderate Transition & $-$4,381 & $-$251\% & $-$1.39$\times$ & B & 3,880 \\
\quad Aggressive Transition & $-$7,964 & $-$375\% & $-$4.37$\times$ & B & 5,635 \\
\midrule
\multicolumn{6}{l}{\textit{Physical Risk Only (CLIMADA)}} \\
\quad Moderate Physical (RCP 4.5) & 1,928 & $-$33\% & 1.58$\times$ & BBB & 475 \\
\quad High Physical (RCP 8.5) & 1,189 & $-$59\% & 1.42$\times$ & BBB & 837 \\
\midrule
\multicolumn{6}{l}{\textit{Combined Risks}} \\
\quad Combined Moderate & $-$5,042 & $-$274\% & $-$1.60$\times$ & B & 4,204 \\
\quad Combined Aggressive & $-$8,411 & $-$390\% & $-$4.32$\times$ & B & 5,854 \\
\bottomrule
\end{tabular}
\vspace{0.5em}

\noindent\textit{Note}: Negative DSCR indicates project cannot service debt (EBITDA $<$ 0). Rating B indicates loss of investment grade status. CRP = Climate Risk Premium.
\end{table}

\subsubsection{Korea Power Plan Dominates Transition Risk}

The most striking finding is that \textit{transition risk dominates physical risk by a factor of 4-6×}. Moderate transition scenarios (based on Korea's power supply plan trajectory) reduce NPV by 251\%, while high physical climate risks (RCP 8.5) reduce NPV by 59\%. This disparity demonstrates that \textbf{government policy—not climate change itself—is the primary driver of coal asset stranding in Korea}.

The mechanism is clear: Korea's 10th Power Supply Plan mandates coal dispatch reductions from 195 TWh (2024) to 130 TWh (2030 NDC target) to 95 TWh (2036). For Samcheok, this translates to capacity factor declines from 60\% (2024) to 45\% (2030) to 32\% (2036) to <10\% by 2045. Revenue collapses proportionally, while fixed costs remain constant, squeezing EBITDA margins and triggering debt service coverage ratio violations.

\subsubsection{CLIMADA Physical Risks Are Localized but Material}

Physical climate risks, while smaller in magnitude than policy risks, are nonetheless material:

\begin{itemize}
    \item \textbf{Wildfire}: 120 km transmission through mountainous terrain increases forced outage rate from 1.2\% (baseline) to 3.0\% (RCP 8.5 2050). The 2022 Uljin-Samcheok mega-fire (213 km² burned) demonstrated this vulnerability empirically.

    \item \textbf{Flood}: Coastal location 2-3 km from East Sea + riverine Osip Creek exposure creates compound flood risk. Annual outage rate increases from 0.2\% to 0.35\% under RCP 8.5 due to monsoon intensification and sea level rise amplification.

    \item \textbf{Sea Level Rise}: Cooling water intake at 5m elevation faces 3\% capacity derating by 2050 under RCP 8.5 (+0.45m SLR) due to reduced hydraulic head and amplified storm surge vulnerability.
\end{itemize}

Combined, CLIMADA hazards reduce effective capacity factor by 5.85\% (RCP 8.5, 2050), translating to \$1.7 billion NPV loss and 837 bps CRP increase under high physical risk scenarios. While less severe than policy impacts, these risks are \textit{unavoidable} (adaptation costly, relocation impossible) and \textit{permanent} (sea level rise irreversible on human timescales).

 \begin{figure}[h]
     \centering
     \includegraphics[width=1.0\textwidth]{data/processed/figures/fig2_waterfall.png}
     \caption{Lifetime Cash Flow Waterfall (11th Plan + Extreme Physical Risk). Revenue is decimated by policy constraints, while fixed costs and financial obligations remain, resulting in negative Net Income.}
     \label{fig:waterfall}
 \end{figure}

\subsection{The Credit Rating Death Spiral}

\subsubsection{Investment Grade Loss at Policy-Driven Thresholds}

A critical non-linearity emerges at the investment grade boundary. All transition scenarios (moderate, aggressive, combined) trigger rating downgrades from investment grade (A/BBB) to non-investment grade (BB/B), while physical-only scenarios maintain investment grade.

The threshold metric is Debt Service Coverage Ratio (DSCR):
\begin{itemize}
    \item \textbf{Baseline}: 1.81× (Investment grade, debt serviceable)
    \item \textbf{Moderate Transition}: -1.39× (B rating, \textit{negative} cash flow, cannot service debt)
    \item \textbf{High Physical}: 1.42× (Investment grade maintained)
\end{itemize}

Transition risk dispatch reductions drive DSCR negative (project becomes insolvent), while CLIMADA physical risks merely reduce positive DSCR. This categorical difference explains the rating cliff.

\subsubsection{Feedback Loop Dynamics}

The credit rating death spiral operates as follows:

\begin{enumerate}
    \item \textbf{T=0}: Plant commissioned with A rating, 6.1\% bond yield (150 bps over risk-free)
    \item \textbf{T=1-5}: Korea Power Plan dispatch reductions reduce revenue 15-20\%/year
    \item \textbf{T=5}: EBITDA/Interest falls below 2.0× → rating downgraded to BBB (250 bps)
    \item \textbf{T=6}: Increased interest expense further reduces DSCR → downgrade to BB (400 bps)
    \item \textbf{T=7}: EBITDA turns negative → downgrade to B (600 bps)
    \item \textbf{T=8}: Refinancing impossible at 600 bps spread; technical default
\end{enumerate}
 
 \begin{figure}[h]
     \centering
     \includegraphics[width=1.0\textwidth]{data/processed/figures/fig3_rating_migration.png}
     \caption{Credit Rating Migration Paths. Policy-driven scenarios (Red/Orange) trigger rapid downgrades below Investment Grade (dotted line) by 2030, while Baseline (Blue) remains stable.}
     \label{fig:rating_migration}
 \end{figure}

At the 600 bps spread level, debt service exceeds \textit{total revenue}, making the project mathematically unfinanceable. The non-linear jump from 150 bps (A) to 600 bps (B) represents a 300\% increase in financing costs, explaining why NPV can exceed -100\% (plant worth less than zero, creating negative value for equity holders).

\subsection{Synergistic Risk Amplification}

Combined scenarios exhibit significant risk effects. The worst-case scenario (aggressive transition + high physical) produces:

\begin{itemize}
    \item \textbf{NPV}: -\$8.4 billion (from \$2.9 billion baseline, a -390\% swing)
    \item \textbf{CRP}: 5,854 bps (58.5 percentage points)
    \item \textbf{DSCR}: -4.32× (deeply negative, unable to service any debt)
\end{itemize}

The combined impact demonstrates risk compounding:
\begin{itemize}
    \item Aggressive transition alone: -375\% NPV change, 5,635 bps CRP
    \item High physical alone: -59\% NPV change, 837 bps CRP
    \item Combined aggressive: -390\% NPV change, 5,854 bps CRP
\end{itemize}

While the incremental impact of physical risks when added to transition risks appears modest (physical adds 219 bps CRP on top of transition's 5,635 bps), this reflects that transition risks already push the project into deep distress. The combined scenario confirms that climate policy creates the dominant stranding risk, with physical hazards providing additional but secondary erosion of value.

\section{Discussion and Policy Implications}

\subsection{Policy as Primary Stranded Asset Driver}

Our most important finding is that \textbf{government policy—specifically Korea's 10th Power Supply Plan—is the dominant driver of coal asset stranding}, not physical climate change or carbon pricing. This has profound implications:

\begin{enumerate}
    \item \textbf{Stranded asset risk is not hypothetical}: Official government commitments (NDC, net-zero 2050) create legally binding dispatch reductions that materially impair coal plant economics. Investors ignoring national energy plans face quantifiable losses.

    \item \textbf{Rating agencies must incorporate policy trajectories}: Current credit ratings (Samcheok bonds trade at A/BBB equivalent, 6.1\% yield) do not reflect power supply plan dispatch constraints. Forward-looking ratings should downgrade based on scheduled capacity factor declines.

    \item \textbf{Early retirement is financially optimal}: Waiting for "natural" economic obsolescence subjects owners to accelerating losses. Negotiated early retirement (with compensation) dominates market-driven collapse.
\end{enumerate}

The "Climate Risk Premium" embedded in Samcheok's 6-7\% bond yields suggests markets are \textit{beginning} to price policy risk, but our analysis indicates CRP should be 3,000-6,000 bps higher under realistic transition scenarios, implying either (a) market complacency or (b) implicit government bailout expectations.

\subsection{Physical Climate Risks Require Location-Specific Quantification}

CLIMADA's spatially-explicit hazard modeling reveals that \textbf{generic physical risk assumptions underestimate site-specific vulnerabilities}:

\begin{itemize}
    \item Samcheok's 120 km mountainous transmission corridors create 2.5× wildfire outage risk vs. flatland plants
    \item Coastal location + riverine exposure creates compound flood risk not captured in national averages
    \item Cooling water intake elevation (5m) determines sea level rise sensitivity more than regional SLR projections
\end{itemize}

This implies that physical risk assessments must be \textit{asset-specific}, not \textit{sector-average}. Insurance pricing, asset valuation, and adaptation investment decisions all require location-explicit hazard data. CLIMADA provides a replicable methodology for Korean coal fleet (60 units across diverse geographies).

\subsection{The Investment Grade Cliff}

The non-linear jump from A (150 bps) to B (600 bps) at the investment grade boundary creates a "cliff risk" that linear models miss. Once EBITDA/Interest falls below 2.0×, rating agencies mechanically downgrade to junk, triggering:

\begin{itemize}
    \item \textbf{Institutional investor exit}: Many pension funds, insurers have mandates prohibiting sub-investment grade holdings
    \item \textbf{Refinancing impossibility}: If bonds mature before 2045, refinancing at 600 bps spreads is mathematically infeasible (debt service > total revenue)
    \item \textbf{Covenant violations}: Project finance agreements typically have rating floor covenants (maintain BBB or better); breach triggers acceleration clauses
\end{itemize}

This suggests that \textbf{preemptive action before crossing the investment grade threshold} is critical. Once rated B, recovery is impossible without extraordinary restructuring (debt forgiveness, government bailout).

\subsection{Just Transition Finance Mechanisms}

If Korea Power Plan dispatch reductions render Samcheok uneconomic by 2040 (10 years early), the losses are:

\begin{itemize}
    \item \textbf{Owner (POSCO)}: \$2-3 billion stranded assets (unamortized capex + negative NPV)
    \item \textbf{Lenders}: \$2 billion debt outstanding (if no prepayment/compensation)
    \item \textbf{Workers}: 300+ direct jobs, 1,000+ indirect in Samcheok region
    \item \textbf{Local government}: Tax revenue loss, community economic disruption
\end{itemize}

Total societal cost: \$4-5 billion + non-monetized social impacts. Market-driven collapse distributes losses chaotically (defaults, job losses, community shocks). Structured transition finance can allocate costs fairly:

\begin{enumerate}
    \item \textbf{Early Retirement Contracts}: Government compensates owner for premature closure (say, 50\% of stranded asset value = \$1-1.5B), avoiding default cascades

    \item \textbf{Transition Bonds}: Issue 10-year bonds backed by carbon auction revenues to fund compensation, worker retraining, community economic diversification

    \item \textbf{Refinancing Facility}: Government guarantees below-investment-grade debt to avoid liquidity crisis, conditional on orderly phase-out agreement

    \item \textbf{Just Transition Fund}: Earmark revenues from coal plant early closure savings (avoided air pollution costs, accelerated renewable deployment) for affected workers/communities
\end{enumerate}

Cost-benefit analysis: \$1.5B government compensation vs. \$5B+ chaotic default costs suggests net savings of \$3.5B from structured transition. Precedents exist (Germany's coal closure compensation, UK coal community support).

\subsection{Implications for Global Coal Finance}

Samcheok is representative of 1,000+ GW coal capacity under construction/planned in developing Asia. If Korea—a wealthy OECD member—faces 50-150\% NPV losses on new coal, similar risks exist in:

\begin{itemize}
    \item \textbf{Vietnam}: 13 GW coal pipeline, NDC targets incompatible with full utilization
    \item \textbf{Indonesia}: 22 GW coal pipeline, net-zero 2060 pledge threatens late-stage plants
    \item \textbf{India}: 25 GW coal pipeline, renewable cost declines eroding competitiveness
\end{itemize}

Combined capital at risk: \$100+ billion. International financial institutions (IFC, ADB, multilateral development banks) financing these projects face material ESG risks. Our methodology (Korea Power Plan integration + CLIMADA physical risks + credit rating migration) is \textbf{replicable and scalable} to assess global coal stranded asset exposure.

\subsection{Limitations and Future Research}

\textbf{Limitations}:
\begin{enumerate}
    \item Korea Power Plan extrapolation beyond 2036 assumes linear decline to net-zero; actual trajectory uncertain
    \item CLIMADA hazard data at ~10 km resolution; site-specific microtopography not captured
    \item Model assumes no adaptation (firebreaks, flood barriers); resilience investments could mitigate physical risks
    \item Static analysis; does not model dynamic dispatch optimization or real options (fuel switching, load following)
\end{enumerate}

\textbf{Future Research}:
\begin{enumerate}
    \item \textbf{Monte Carlo simulation}: Probabilistic Korea Power Plan scenarios + stochastic CLIMADA hazard sampling for VaR/CVaR metrics
    \item \textbf{Portfolio analysis}: Extend to entire Korean coal fleet (60 units, 36 GW) for system-level stranded asset risk
    \item \textbf{Real options}: Value of operational flexibility (early retirement option, fuel switching optionality) vs. committed operation
    \item \textbf{Comparative analysis}: Apply framework to renewables (wind/solar) to demonstrate \textit{negative} CRP (climate-aligned assets have lower cost of capital)
\end{enumerate}

\section{Conclusion}

The Samcheok Blue Power plant, commissioned in 2024 as South Korea's last coal project, exemplifies the financial risks of new fossil infrastructure in a climate-constrained world. Our integrated framework—combining Korea's National Power Supply Plan, CLIMADA physical hazard data, and KIS credit rating methodology—quantifies three key findings:

\begin{enumerate}
    \item \textbf{Policy risk dominates physical risk}: Transition scenarios reduce NPV by 251-375\%, while CLIMADA climate hazards reduce NPV by 33-59\%. Government energy policy—not climate change itself—is the primary driver of coal asset stranding in Korea.

    \item \textbf{Investment grade loss is inevitable}: All transition scenarios (moderate, aggressive, combined) trigger credit rating downgrades from investment grade to non-investment grade, with DSCR falling from 1.81× to negative values. Once DSCR falls below 1.0×, recovery is impossible without extraordinary intervention.

    \item \textbf{Climate Risk Premium is material and quantifiable}: The CRP ranges from 475 bps (moderate physical risk) to 5,854 bps (combined aggressive scenario), making projects economically unviable. Samcheok's current 6-7\% bond yields suggest markets are beginning to price policy risk but remain complacent about downside scenarios.
\end{enumerate}

These findings have urgent policy implications. Market forces \textit{will} accelerate coal plant closures faster than policy mandates, but disorderly exits pose financial stability risks. Structured "Just Transition Finance" mechanisms—early retirement contracts, transition bonds, refinancing facilities—can allocate costs fairly while achieving decarbonization targets. The alternative—chaotic defaults, job losses, community disruption—is economically and socially costlier.

For investors, the message is clear: coal assets face existential risks from climate policy, physical hazards, and credit rating migration. The "Climate Risk Premium" is not theoretical; it is material, quantifiable, and accelerating. Ignoring government energy plans and spatially-explicit climate science invites massive capital destruction.

For policymakers, Samcheok demonstrates that climate commitments have financial consequences. National energy plans are not aspirational; they are legally binding constraints that destroy asset value. Recognizing stranded asset risks explicitly—and designing transition finance mechanisms proactively—is essential for orderly, just, and economically efficient decarbonization.

\section*{Acknowledgments}
This research was supported by PLANiT Institute. We thank Solutions for Our Climate (SFOC) for data on Korean coal policies and the CLIMADA development team at ETH Zurich for open-source hazard modeling tools.

\section*{References}
\begin{enumerate}
    \item Ministry of Trade, Industry and Energy (MOTIE). (2023). \textit{10th Basic Plan for Long-term Electricity Supply and Demand (2023--2036)}. Seoul: Government of the Republic of Korea.

    \item Bresch, D. N., \& Aznar-Siguan, G. (2021). CLIMADA v1.4.1: Towards a globally consistent adaptation options appraisal tool. \textit{Geoscientific Model Development}, 14(5), 3085--3097. https://doi.org/10.5194/gmd-14-3085-2021

    \item Korea Investors Service (KIS). (2020). \textit{Credit Rating Methodology for Private Power Generation (IPP) Sector}. Seoul: KIS.

    \item IPCC. (2021). Regional Sea Level Change (Chapter 9). In \textit{Climate Change 2021: The Physical Science Basis. Contribution of Working Group I to the Sixth Assessment Report of the Intergovernmental Panel on Climate Change} (pp. 1211--1362). Cambridge University Press.

    \item Solutions for Our Climate (SFOC). (2022). \textit{Roadmap for Coal Phase-out in South Korea: Analysis and Policy Recommendations}. Seoul: SFOC.

    \item Carbon Tracker Initiative. (2021). \textit{Stranded Assets and Thermal Coal: An Analysis of Environment-Related Risks}. London: Carbon Tracker.

    \item Network for Greening the Financial System (NGFS). (2022). \textit{NGFS Climate Scenarios for Central Banks and Supervisors} (3rd ed.). Paris: NGFS Secretariat.

    \item Korea Power Exchange (KPX). (2024). \textit{Electricity Market Statistics 2024}. Seoul: KPX.

    \item International Energy Agency (IEA). (2023). \textit{World Energy Outlook 2023}. Paris: IEA.

    \item Caldecott, B., Howarth, N., \& McSharry, P. (2013). Stranded Assets in Agriculture: Protecting Value from Environment-Related Risks. \textit{Smith School of Enterprise and the Environment Working Paper}.

    \item Battiston, S., Mandel, A., Monasterolo, I., Sch{\"u}tze, F., \& Visentin, G. (2017). A climate stress-test of the financial system. \textit{Nature Climate Change}, 7(4), 283--288.

    \item Dietz, S., Bowen, A., Dixon, C., \& Gradwell, P. (2016). `Climate value at risk' of global financial assets. \textit{Nature Climate Change}, 6(7), 676--679.

    \item Korea Meteorological Administration (KMA). (2020). \textit{Korean Climate Change Assessment Report 2020: Climate Impact and Adaptation}. Seoul: KMA.

    \item Global Flood Awareness System (GloFAS). (2023). \textit{GloFAS River Discharge Reanalysis v4.0}. European Commission Joint Research Centre. https://doi.org/10.24381/cds.a4fdd6b9

    \item Presidential Commission on Carbon Neutrality and Green Growth. (2021). \textit{2050 Carbon Neutral Scenario}. Seoul: Government of the Republic of Korea.
\end{enumerate}

\newpage
\appendix

\section{Samcheok Blue Power Plant Financial Parameters}
\label{app:parameters}

\begin{table}[H]
\centering
\caption{Key Financial Parameters Used in Analysis}
\small
\begin{tabular}{lrl}
\toprule
\textbf{Parameter} & \textbf{Value} & \textbf{Source} \\
\midrule
\multicolumn{3}{l}{\textit{Plant Specifications}} \\
Capacity & 2,100 MW & Public filings \\
Technology & USC Coal & Plant specifications \\
Investment Cost & \$4.9 billion & POSCO disclosures \\
Design Life & 30 years & Industry standard \\
\midrule
\multicolumn{3}{l}{\textit{Financial Structure}} \\
Debt/Equity Ratio & 70\%/30\% & Project financing terms \\
Cost of Debt & 6.1--7.4\% & Bond market data \\
Cost of Equity & 10\% & Industry benchmark \\
Corporate Tax Rate & 24\% & Korean tax law \\
Depreciation & Straight-line, 30 yr & Accounting standards \\
\midrule
\multicolumn{3}{l}{\textit{Operating Assumptions}} \\
Baseline Capacity Factor & 50\% & Conservative estimate \\
Electricity Price & \$80/MWh & KPX market data \\
Fixed O\&M & \$50,000/MW/yr & Industry benchmarks \\
Variable O\&M & \$5/MWh & Industry benchmarks \\
Heat Rate & 8,500 BTU/kWh & USC efficiency \\
Coal Price & \$100/tonne & Market data \\
\midrule
\multicolumn{3}{l}{\textit{Risk-Free Rate}} \\
10-Year KTB Yield & 3.0\% & Bank of Korea \\
\bottomrule
\end{tabular}
\end{table}

\section{Physical Hazard Parameters}
\label{app:climada}

\begin{table}[H]
\centering
\caption{Physical Hazard Impact Parameters for Samcheok Location (37.44°N, 129.17°E)}
\small
\begin{tabular}{lcccl}
\toprule
\textbf{Hazard} & \textbf{Baseline} & \textbf{RCP 4.5 (2050)} & \textbf{RCP 8.5 (2050)} & \textbf{Data Source} \\
\midrule
\multicolumn{5}{l}{\textit{Wildfire (Transmission Outage Rate)}} \\
Fire Weather Index & 18.5 & 26.2 & 35.8 & KMA/ERA5-Land \\
Annual Outage Rate & 1.20\% & 1.66\% & 2.24\% & Model \\
\midrule
\multicolumn{5}{l}{\textit{Flood (Combined Riverine + Coastal)}} \\
Annual Outage Rate & 0.14\% & 0.21\% & 0.24\% & GLOFAS/K-water \\
\midrule
\multicolumn{5}{l}{\textit{Sea Level Rise (Capacity Derate)}} \\
SLR Projection & 0 m & +0.24 m & +0.32 m & IPCC AR6 Ch.9 \\
Capacity Derate & 0\% & 2.4\% & 3.2\% & Model \\
\midrule
\textbf{Total CF Impact} & \textbf{1.34\%} & \textbf{4.27\%} & \textbf{5.68\%} & \\
\bottomrule
\end{tabular}
\vspace{0.5em}

\noindent\textit{Sources}: Fire Weather Index from KMA Climate Change Assessment Report 2020 and ERA5-Land reanalysis; Flood data from GLOFAS v3.1 and K-water flood hazard maps; Sea level rise from IPCC AR6 WG1 Chapter 9, Table 9.9 for East Asian Marginal Seas.
\end{table}

\section{Korea Power Supply Plan Dispatch Trajectory}
\label{app:power_plan}

\begin{table}[H]
\centering
\caption{Coal Generation Trajectory from 10th Power Supply Plan}
\small
\begin{tabular}{rrrrl}
\toprule
\textbf{Year} & \textbf{Coal (TWh)} & \textbf{Total (TWh)} & \textbf{Coal Share} & \textbf{Implied CF (2.1 GW)} \\
\midrule
2024 & 195 & 600 & 32.5\% & 65\% \\
2025 & 185 & 615 & 30.1\% & 60\% \\
2028 & 160 & 655 & 24.4\% & 52\% \\
2030 & 130 & 675 & 19.3\% & 45\% (NDC) \\
2034 & 110 & 710 & 15.5\% & 38\% \\
2036 & 95 & 735 & 12.9\% & 32\% (Plan) \\
2040 & 70 & 780 & 9.0\% & 22\% \\
2045 & 40 & 820 & 4.9\% & 12\% \\
2050 & 15 & 860 & 1.7\% & 4\% (Net-Zero) \\
\bottomrule
\end{tabular}
\vspace{0.5em}

\noindent\textit{Source}: MOTIE 10th Basic Plan (2023--2036), extrapolated to 2050 based on NDC and net-zero commitments.
\end{table}

\end{document}
