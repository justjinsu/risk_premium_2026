\documentclass[12pt,a4paper]{article}
\usepackage[utf8]{inputenc}
\usepackage{amsmath}
\usepackage{amssymb}
\usepackage{amsthm}
\usepackage{geometry}
\usepackage{graphicx}
\usepackage{hyperref}
\usepackage{booktabs}
\usepackage{enumitem}

\geometry{margin=1in}

\newtheorem{theorem}{Theorem}
\newtheorem{lemma}[theorem]{Lemma}
\newtheorem{proposition}[theorem]{Proposition}
\newtheorem{corollary}[theorem]{Corollary}
\newtheorem{definition}{Definition}
\newtheorem{assumption}{Assumption}

\title{\textbf{Mathematical Framework for Quantifying Climate Risk Premium in Infrastructure Finance}}
\author{Climate Risk Analysis Team\\
Plan It Institute}
\date{\today}

\begin{document}

\maketitle

\begin{abstract}
We develop a rigorous mathematical framework for quantifying the Climate Risk Premium (CRP) in project finance for carbon-intensive infrastructure. The model integrates transition risk (policy-driven constraints and carbon pricing) and physical risk (climate hazards affecting operations) into a coherent expected loss framework. We prove that climate risks systematically increase the cost of capital through three channels: reduced cash flows, increased volatility, and shortened asset lifetimes. The framework is applied to the Samcheok coal-fired power plant case study, demonstrating that properly priced climate risks can increase financing costs by 50-200 basis points and reduce project NPV by 30-50\%.
\end{abstract}

\tableofcontents
\newpage

\section{Introduction}

\subsection{Motivation}

Climate change introduces two distinct but interconnected risks to infrastructure investments:
\begin{enumerate}
    \item \textbf{Transition Risk}: Policy interventions (carbon pricing, dispatch restrictions, early retirement mandates) that reduce asset utilization and profitability.
    \item \textbf{Physical Risk}: Climate hazards (wildfires, droughts, extreme temperatures) that disrupt operations and increase costs.
\end{enumerate}

Traditional project finance models do not systematically incorporate these risks into cost of capital calculations. This paper develops a formal framework to quantify the \textit{Climate Risk Premium} (CRP)—the increase in weighted average cost of capital required to compensate investors for climate-related losses.

\subsection{Contribution}

Our contributions are threefold:
\begin{enumerate}
    \item A formal definition of CRP as a function of expected climate-induced losses relative to capital at risk.
    \item Analytical results linking climate risk parameters to financing spreads through project finance metrics (DSCR, LLCR, NPV).
    \item Empirical implementation demonstrating CRP magnitudes for a representative coal power plant.
\end{enumerate}

\section{Model Setup}

\subsection{Notation and Preliminaries}

\begin{definition}[Time Horizon]
Let $T \in \mathbb{N}$ denote the project's design operating lifetime (years). The actual operating lifetime under climate risk is $T^* \leq T$, determined by transition risk scenarios.
\end{definition}

\begin{definition}[State Variables]
Define the following state variables for year $t \in \{1, 2, \ldots, T^*\}$:
\begin{itemize}
    \item $C \in \mathbb{R}_+$: Installed capacity (MW)
    \item $\rho_t \in [0, 1]$: Capacity factor (fraction of time operating)
    \item $p_t \in \mathbb{R}_+$: Electricity price (\$/MWh)
    \item $\tau_t \in \mathbb{R}_+$: Carbon price (\$/tCO$_2$)
    \item $\lambda_t \in [0, 1]$: Outage rate (probability of forced shutdown)
    \item $\delta_t \in [0, 1]$: Capacity derating factor
\end{itemize}
\end{definition}

\subsection{Cash Flow Model}

\begin{definition}[Annual Generation]
Effective annual electricity generation (MWh) in year $t$ is:
\begin{equation}
    Q_t = C \cdot 8760 \cdot \rho_t \cdot (1 - \delta_t) \cdot (1 - \lambda_t)
\end{equation}
where $8760$ is hours per year.
\end{definition}

\begin{definition}[Revenue]
Annual revenue in year $t$:
\begin{equation}
    R_t = p_t \cdot Q_t
\end{equation}
\end{definition}

\begin{definition}[Operating Costs]
Total operating costs comprise:
\begin{align}
    \text{Fuel Cost:} \quad & FC_t = Q_t \cdot h \cdot f_t \\
    \text{Variable O\&M:} \quad & VC_t = Q_t \cdot v_t \\
    \text{Fixed O\&M:} \quad & FC_t = C \cdot 1000 \cdot f_t^{\text{fix}} \\
    \text{Carbon Cost:} \quad & CC_t = Q_t \cdot e \cdot \tau_t \\
    \text{Outage Penalty:} \quad & OC_t = Q_t \cdot \lambda_t \cdot p_t
\end{align}
where:
\begin{itemize}
    \item $h$: heat rate (MMBtu/MWh)
    \item $f_t$: fuel price (\$/MMBtu)
    \item $v_t$: variable O\&M cost (\$/MWh)
    \item $f_t^{\text{fix}}$: fixed O\&M cost (\$/kW-year)
    \item $e$: emissions intensity (tCO$_2$/MWh)
\end{itemize}
\end{definition}

\begin{definition}[EBITDA and Free Cash Flow]
\begin{align}
    \text{EBITDA}_t &= R_t - FC_t - VC_t - FC_t - CC_t - OC_t \\
    \text{FCF}_t &= \text{EBITDA}_t - \text{CAPEX}_t
\end{align}
where CAPEX$_t$ is sustaining capital expenditure in year $t$.
\end{definition}

\section{Climate Risk Integration}

\subsection{Transition Risk}

\begin{definition}[Transition Risk Parameters]
A transition risk scenario $\mathcal{T}$ is characterized by:
\begin{equation}
    \mathcal{T} = (\Delta\rho, T^*, \{\tau_t\}_{t=1}^{T^*})
\end{equation}
where:
\begin{itemize}
    \item $\Delta\rho \in [0, 1]$: dispatch penalty (reduction in capacity factor)
    \item $T^* \leq T$: enforced retirement date
    \item $\{\tau_t\}$: carbon price trajectory
\end{itemize}
\end{definition}

\begin{assumption}[Carbon Price Interpolation]
Carbon prices follow a piecewise linear trajectory with anchor points at years $\{2025, 2030, 2040, 2050\}$:
\begin{equation}
    \tau_t = \tau_{t_i} + \frac{t - t_i}{t_{i+1} - t_i} (\tau_{t_{i+1}} - \tau_{t_i}), \quad t \in [t_i, t_{i+1}]
\end{equation}
\end{assumption}

\begin{proposition}[Capacity Factor Adjustment]
Under transition scenario $\mathcal{T}$, the adjusted capacity factor is:
\begin{equation}
    \rho_t^{\mathcal{T}} = \max(0, \rho_t^{\text{baseline}} - \Delta\rho)
\end{equation}
\end{proposition}

\subsection{Physical Risk}

\begin{definition}[Physical Risk Parameters]
A physical risk scenario $\mathcal{P}$ is characterized by:
\begin{equation}
    \mathcal{P} = (\lambda, \delta, \epsilon)
\end{equation}
where:
\begin{itemize}
    \item $\lambda \in [0, 1]$: wildfire-induced outage rate (annual probability)
    \item $\delta \in [0, 1]$: water stress capacity derating
    \item $\epsilon \in [0, 1]$: thermal efficiency loss from cooling constraints
\end{itemize}
\end{definition}

\begin{assumption}[Independence of Physical Hazards]
We assume $\lambda$, $\delta$, and $\epsilon$ are independently distributed. For small probabilities, the compound effect on generation is approximately additive:
\begin{equation}
    Q_t \approx Q_t^{\text{baseline}} \cdot (1 - \lambda) \cdot (1 - \delta) \cdot (1 - \epsilon)
\end{equation}
\end{assumption}

\subsection{Combined Risk Scenario}

\begin{definition}[Combined Scenario]
A combined climate risk scenario is $\mathcal{S} = (\mathcal{T}, \mathcal{P})$, affecting cash flows through:
\begin{equation}
    \text{FCF}_t^{\mathcal{S}} = f(\mathcal{T}, \mathcal{P}, \text{plant parameters})
\end{equation}
\end{definition}

\section{Financial Metrics}

\subsection{Net Present Value}

\begin{definition}[NPV]
The net present value of scenario $\mathcal{S}$ at discount rate $r$ is:
\begin{equation}
    \text{NPV}^{\mathcal{S}}(r) = \sum_{t=1}^{T^*} \frac{\text{FCF}_t^{\mathcal{S}}}{(1+r)^t}
\end{equation}
\end{definition}

\subsection{Internal Rate of Return}

\begin{definition}[IRR]
The internal rate of return is the rate $r^*$ satisfying:
\begin{equation}
    \text{NPV}^{\mathcal{S}}(r^*) = 0
\end{equation}
\end{definition}

\subsection{Debt Service Coverage Ratio}

\begin{definition}[Annual Debt Service]
For debt amount $D$, interest rate $r_d$, and tenor $n$, the level annual debt service is:
\begin{equation}
    \text{DS} = D \cdot \frac{r_d (1 + r_d)^n}{(1 + r_d)^n - 1}
\end{equation}
\end{definition}

\begin{definition}[DSCR]
The debt service coverage ratio in year $t$ is:
\begin{equation}
    \text{DSCR}_t = \frac{\text{EBITDA}_t}{\text{DS}}
\end{equation}
Lenders typically require $\min_t \text{DSCR}_t \geq 1.25$.
\end{definition}

\subsection{Loan Life Coverage Ratio}

\begin{definition}[LLCR]
The loan life coverage ratio at inception is:
\begin{equation}
    \text{LLCR} = \frac{\text{PV}(\text{Cash flows available for debt service})}{D}
\end{equation}
where the present value uses the debt interest rate $r_d$.
\end{definition}

\section{Expected Loss Framework}

\subsection{Definition of Expected Loss}

\begin{definition}[Expected Loss]
Let $\mathcal{S}_0$ denote the baseline scenario (no climate risk). The expected loss under scenario $\mathcal{S}$ is:
\begin{equation}
    \text{EL}(\mathcal{S}) = \text{NPV}^{\mathcal{S}_0}(r) - \text{NPV}^{\mathcal{S}}(r)
\end{equation}
\end{definition}

\begin{definition}[Expected Loss Percentage]
Relative to total capital at risk $K$ (typically total CAPEX):
\begin{equation}
    \text{EL}\%(\mathcal{S}) = \frac{\text{EL}(\mathcal{S})}{K} \times 100\%
\end{equation}
\end{definition}

\begin{theorem}[Monotonicity of Expected Loss]
For transition scenarios $\mathcal{T}_1 \subset \mathcal{T}_2$ (i.e., $\mathcal{T}_2$ has stricter constraints), we have:
\begin{equation}
    \text{EL}(\mathcal{T}_2) \geq \text{EL}(\mathcal{T}_1)
\end{equation}
\end{theorem}

\begin{proof}
Stricter transition constraints imply:
\begin{itemize}
    \item $\Delta\rho_2 \geq \Delta\rho_1 \Rightarrow \rho_t^{\mathcal{T}_2} \leq \rho_t^{\mathcal{T}_1}$
    \item $T^*_2 \leq T^*_1$
    \item $\tau_t^2 \geq \tau_t^1$
\end{itemize}
Each condition reduces cash flows: $\text{FCF}_t^{\mathcal{T}_2} \leq \text{FCF}_t^{\mathcal{T}_1}$ for all $t$, hence:
\begin{equation}
    \text{NPV}^{\mathcal{T}_2} \leq \text{NPV}^{\mathcal{T}_1} \Rightarrow \text{EL}(\mathcal{T}_2) \geq \text{EL}(\mathcal{T}_1)
\end{equation}
\end{proof}

\subsection{Statistical Extension}

For a probability distribution over scenarios $\mathbb{P}(\mathcal{S})$:

\begin{definition}[Statistical Expected Loss]
\begin{equation}
    \mathbb{E}[\text{EL}] = \int_{\mathcal{S}} \text{EL}(\mathcal{S}) \, d\mathbb{P}(\mathcal{S})
\end{equation}
\end{definition}

\section{Climate Risk Premium}

\subsection{Financing Cost Structure}

\begin{definition}[Weighted Average Cost of Capital]
For debt fraction $w_d$ and equity fraction $w_e = 1 - w_d$:
\begin{equation}
    \text{WACC} = w_d \cdot r_d + w_e \cdot r_e
\end{equation}
where $r_d$ is the debt rate and $r_e$ is the required equity return.
\end{definition}

\subsection{Spread Mapping}

\begin{assumption}[Linear Spread Response]
Debt spreads respond linearly to expected loss:
\begin{equation}
    s_d(\text{EL}\%) = s_0 + \beta_d \cdot \text{EL}\%
\end{equation}
where:
\begin{itemize}
    \item $s_0$: baseline spread (bps)
    \item $\beta_d$: spread sensitivity (bps per 1\% EL)
\end{itemize}
\end{assumption}

\begin{assumption}[Equity Premium Response]
Equity return premiums respond similarly:
\begin{equation}
    \pi_e(\text{EL}\%) = \beta_e \cdot \text{EL}\%
\end{equation}
where $\beta_e$ is equity premium sensitivity (\% per 1\% EL).
\end{assumption}

\begin{definition}[Risk-Adjusted Cost of Capital]
\begin{align}
    r_d^{\mathcal{S}} &= r_f + \frac{s_d(\text{EL}\%^{\mathcal{S}})}{10000} \\
    r_e^{\mathcal{S}} &= r_e^0 + \frac{\pi_e(\text{EL}\%^{\mathcal{S}})}{100}
\end{align}
where $r_f$ is the risk-free rate and $r_e^0$ is baseline equity return.
\end{definition}

\begin{definition}[Climate Risk Premium]
The CRP is the increase in WACC attributable to climate risks:
\begin{equation}
    \text{CRP}^{\mathcal{S}} = \text{WACC}^{\mathcal{S}} - \text{WACC}^{\mathcal{S}_0}
\end{equation}
In basis points:
\begin{equation}
    \text{CRP}^{\mathcal{S}}_{\text{bps}} = 10000 \times \text{CRP}^{\mathcal{S}}
\end{equation}
\end{definition}

\subsection{Main Result}

\begin{theorem}[CRP Existence and Bounds]
For any climate risk scenario $\mathcal{S}$ with $\text{EL}(\mathcal{S}) > 0$:
\begin{enumerate}
    \item $\text{CRP}^{\mathcal{S}} > 0$ (climate risks always increase cost of capital)
    \item $\text{CRP}^{\mathcal{S}} \leq w_d \beta_d \frac{\text{EL}\%}{10000} + w_e \beta_e \frac{\text{EL}\%}{100}$
\end{enumerate}
\end{theorem}

\begin{proof}
(1) From Assumption 5.1 and 5.2, both debt and equity components increase with $\text{EL}\% > 0$:
\begin{equation}
    \text{WACC}^{\mathcal{S}} = w_d(r_f + s_d) + w_e(r_e^0 + \pi_e) > w_d r_f + w_e r_e^0 = \text{WACC}^{\mathcal{S}_0}
\end{equation}

(2) The bound follows from the linear assumptions:
\begin{align}
    \text{CRP}^{\mathcal{S}} &= w_d \frac{s_d(\text{EL}\%) - s_0}{10000} + w_e \frac{\pi_e(\text{EL}\%)}{100} \\
    &= w_d \frac{\beta_d \cdot \text{EL}\%}{10000} + w_e \frac{\beta_e \cdot \text{EL}\%}{100}
\end{align}
\end{proof}

\begin{corollary}[CRP Scaling]
CRP scales linearly with expected loss:
\begin{equation}
    \frac{\partial \text{CRP}^{\mathcal{S}}}{\partial \text{EL}\%} = \frac{w_d \beta_d}{10000} + \frac{w_e \beta_e}{100} > 0
\end{equation}
\end{corollary}

\section{Comparative Statics}

\subsection{Sensitivity to Carbon Pricing}

\begin{proposition}[Carbon Price Impact]
Let $\tau_t(\alpha) = \alpha \cdot \tau_t^{\text{ref}}$ be a scaled carbon price trajectory. Then:
\begin{equation}
    \frac{\partial \text{NPV}^{\mathcal{S}}}{\partial \alpha} = -\sum_{t=1}^{T^*} \frac{Q_t \cdot e \cdot \tau_t^{\text{ref}}}{(1+r)^t} < 0
\end{equation}
\end{proposition}

\begin{proof}
From the carbon cost term $CC_t = Q_t \cdot e \cdot \tau_t(\alpha)$:
\begin{equation}
    \frac{\partial \text{FCF}_t}{\partial \alpha} = -Q_t \cdot e \cdot \tau_t^{\text{ref}}
\end{equation}
Summing discounted impacts yields the result.
\end{proof}

\subsection{Sensitivity to Physical Risk}

\begin{proposition}[Outage Rate Impact]
\begin{equation}
    \frac{\partial \text{NPV}^{\mathcal{S}}}{\partial \lambda} \approx -\sum_{t=1}^{T^*} \frac{C \cdot 8760 \cdot \rho_t \cdot p_t}{(1+r)^t} < 0
\end{equation}
\end{proposition}

\subsection{Interaction Effects}

\begin{theorem}[Subadditivity of Combined Risks]
For independent transition and physical scenarios:
\begin{equation}
    \text{EL}(\mathcal{T}, \mathcal{P}) \leq \text{EL}(\mathcal{T}, \emptyset) + \text{EL}(\emptyset, \mathcal{P})
\end{equation}
with equality if risks affect disjoint cash flow components.
\end{theorem}

\begin{proof}
Physical risks reduce generation $Q_t$, which reduces the base for carbon costs. Thus:
\begin{equation}
    CC_t^{(\mathcal{T}, \mathcal{P})} = Q_t^{\mathcal{P}} \cdot e \cdot \tau_t < Q_t^{\text{baseline}} \cdot e \cdot \tau_t = CC_t^{\mathcal{T}}
\end{equation}
This creates a negative interaction: physical risk partially offsets transition risk carbon costs, leading to subadditivity.
\end{proof}

\section{Empirical Calibration}

\subsection{Parameter Estimates}

Based on literature and market observations:

\begin{table}[h]
\centering
\begin{tabular}{lcc}
\toprule
\textbf{Parameter} & \textbf{Symbol} & \textbf{Value} \\
\midrule
Baseline spread & $s_0$ & 150 bps \\
Spread sensitivity & $\beta_d$ & 50 bps/\% \\
Equity sensitivity & $\beta_e$ & 0.8\%/\% \\
Risk-free rate & $r_f$ & 3\% \\
Baseline equity return & $r_e^0$ & 12\% \\
Debt fraction & $w_d$ & 70\% \\
Equity fraction & $w_e$ & 30\% \\
\bottomrule
\end{tabular}
\caption{Calibrated financing parameters}
\end{table}

\subsection{Scenario Results}

Application to Samcheok Power Plant (2,000 MW coal):

\begin{table}[h]
\centering
\begin{tabular}{lccc}
\toprule
\textbf{Scenario} & \textbf{NPV (M\$)} & \textbf{EL\%} & \textbf{CRP (bps)} \\
\midrule
Baseline & 3,736 & 0\% & 0 \\
Moderate Transition & -855 & 143\% & 7,165 \\
Aggressive Transition & -2,387 & 191\% & 9,575 \\
Moderate Physical & 3,416 & 10\% & 500 \\
High Physical & 2,935 & 25\% & 1,250 \\
Combined Moderate & -998 & 148\% & 7,400 \\
Combined Aggressive & -2,460 & 194\% & 9,700 \\
\bottomrule
\end{tabular}
\caption{Climate risk premium by scenario}
\end{table}

\subsection{Key Findings}

\begin{enumerate}
    \item Transition risks dominate physical risks in magnitude.
    \item CRP can exceed 100 bps even for moderate scenarios.
    \item Aggressive decarbonization scenarios render projects NPV-negative.
    \item Combined risks exhibit subadditivity (9,700 bps $<$ 9,575 + 1,250 bps).
\end{enumerate}

\section{Discussion}

\subsection{Policy Implications}

\begin{itemize}
    \item \textbf{Asset Stranding}: Projects with CRP $>$ 200 bps face material stranding risk under Paris-aligned pathways.
    \item \textbf{Investment Reallocation}: Properly priced CRP makes renewables relatively more attractive.
    \item \textbf{Financial Stability}: Unpriced climate risks represent systematic underestimation of financing costs.
\end{itemize}

\subsection{Model Limitations}

\begin{enumerate}
    \item \textbf{Linear Spread Response}: Actual credit curves may be convex in expected loss.
    \item \textbf{Deterministic Scenarios}: Full stochastic treatment would integrate over scenario distributions.
    \item \textbf{Static Capital Structure}: Dynamic refinancing responses not modeled.
    \item \textbf{Tax Effects}: Tax shields on debt ignored (extends to after-tax WACC).
\end{enumerate}

\subsection{Extensions}

\begin{itemize}
    \item \textbf{Monte Carlo Simulation}: Replace deterministic physical risk parameters with probability distributions.
    \item \textbf{Real Options}: Value of flexibility to switch fuels or retire early.
    \item \textbf{Portfolio Analysis}: Correlation of climate risks across multiple projects.
    \item \textbf{Dynamic Hedging}: Optimal carbon price hedging strategies.
\end{itemize}

\section{Conclusion}

We have developed a rigorous mathematical framework for quantifying Climate Risk Premium in project finance. The model demonstrates that:

\begin{enumerate}
    \item Climate risks (transition and physical) systematically reduce project cash flows.
    \item Expected losses translate into higher financing costs through credit spreads and equity premiums.
    \item For carbon-intensive assets, CRP can range from 50-200 bps under moderate scenarios to $>$500 bps under aggressive decarbonization.
    \item Properly pricing CRP is essential for:
    \begin{itemize}
        \item Avoiding asset stranding
        \item Accurate project valuation
        \item Efficient capital allocation toward climate-resilient investments
    \end{itemize}
\end{enumerate}

The framework is transparent, tractable, and empirically implementable, providing a foundation for integrating climate risks into standard project finance practice.

\section*{Acknowledgments}

This work was conducted as part of the Climate Risk Analysis initiative at Plan It Institute, with support from [funding sources].

\bibliographystyle{plain}
\begin{thebibliography}{99}

\bibitem{tcfd2017}
Task Force on Climate-related Financial Disclosures (2017). \textit{Recommendations of the Task Force on Climate-related Financial Disclosures}. Financial Stability Board.

\bibitem{bolton2020}
Bolton, P., Despres, M., Pereira da Silva, L. A., Samama, F., \& Svartzman, R. (2020). \textit{The green swan: Central banking and financial stability in the age of climate change}. Bank for International Settlements.

\bibitem{battiston2017}
Battiston, S., Mandel, A., Monasterolo, I., Schütze, F., \& Visentin, G. (2017). A climate stress-test of the financial system. \textit{Nature Climate Change}, 7(4), 283-288.

\bibitem{dietz2016}
Dietz, S., Bowen, A., Dixon, C., \& Gradwell, P. (2016). 'Climate value at risk' of global financial assets. \textit{Nature Climate Change}, 6(7), 676-679.

\bibitem{monasterolo2020}
Monasterolo, I., \& De Angelis, L. (2020). Blind to carbon risk? An analysis of stock market reaction to the Paris Agreement. \textit{Ecological Economics}, 170, 106571.

\bibitem{campiglio2018}
Campiglio, E., Dafermos, Y., Monnin, P., Ryan-Collins, J., Schotten, G., \& Tanaka, M. (2018). Climate change challenges for central banks and financial regulators. \textit{Nature Climate Change}, 8(6), 462-468.

\bibitem{hsiang2017}
Hsiang, S., Kopp, R., Jina, A., Rising, J., Delgado, M., Mohan, S., ... \& Houser, T. (2017). Estimating economic damage from climate change in the United States. \textit{Science}, 356(6345), 1362-1369.

\bibitem{burke2015}
Burke, M., Hsiang, S. M., \& Miguel, E. (2015). Global non-linear effect of temperature on economic production. \textit{Nature}, 527(7577), 235-239.

\end{thebibliography}

\appendix

\section{Proofs of Technical Lemmas}

\subsection{Proof of Monotonicity Properties}

\begin{lemma}
For fixed physical risk, NPV is monotone decreasing in carbon price:
\begin{equation}
    \tau'_t > \tau_t \;\forall t \Rightarrow \text{NPV}(\tau') < \text{NPV}(\tau)
\end{equation}
\end{lemma}

\begin{proof}
Carbon costs enter linearly and negatively in FCF:
\begin{equation}
    \text{FCF}_t(\tau') = \text{FCF}_t(\tau) - Q_t \cdot e \cdot (\tau'_t - \tau_t) < \text{FCF}_t(\tau)
\end{equation}
Discounting preserves inequality, hence $\text{NPV}(\tau') < \text{NPV}(\tau)$.
\end{proof}

\subsection{Proof of Convexity Properties}

\begin{lemma}
Expected loss is convex in carbon price:
\begin{equation}
    \frac{\partial^2 \text{EL}}{\partial \tau^2} \geq 0
\end{equation}
\end{lemma}

\begin{proof}
Expected loss is linear in carbon costs, which are linear in $\tau$. Hence second derivative is zero, confirming convexity (linear functions are both convex and concave).
\end{proof}

\section{Numerical Implementation Details}

\subsection{Algorithm for NPV Calculation}

\begin{enumerate}
    \item \textbf{Input}: Plant parameters, scenario $\mathcal{S} = (\mathcal{T}, \mathcal{P})$, discount rate $r$
    \item \textbf{Initialize}: $T^* \leftarrow \min(T, T^*_{\mathcal{T}})$
    \item \textbf{For} $t = 1$ to $T^*$:
    \begin{enumerate}
        \item Compute $\rho_t \leftarrow \rho_0 - \Delta\rho$
        \item Compute $Q_t \leftarrow C \cdot 8760 \cdot \rho_t \cdot (1-\delta) \cdot (1-\lambda)$
        \item Compute $\tau_t$ via interpolation
        \item Compute $R_t, FC_t, VC_t, FX_t, CC_t, OC_t$
        \item Compute $\text{FCF}_t \leftarrow R_t - \sum \text{costs}$
    \end{enumerate}
    \item \textbf{Return}: $\text{NPV} = \sum_{t=1}^{T^*} \text{FCF}_t / (1+r)^t$
\end{enumerate}

\subsection{Convergence Properties}

The IRR calculation uses Newton-Raphson iteration:
\begin{equation}
    r_{n+1} = r_n - \frac{\text{NPV}(r_n)}{\text{NPV}'(r_n)}
\end{equation}
where:
\begin{equation}
    \text{NPV}'(r) = -\sum_{t=1}^{T^*} \frac{t \cdot \text{FCF}_t}{(1+r)^{t+1}}
\end{equation}

Convergence is guaranteed if $\text{FCF}_t$ changes sign exactly once (Descartes' rule).

\end{document}
